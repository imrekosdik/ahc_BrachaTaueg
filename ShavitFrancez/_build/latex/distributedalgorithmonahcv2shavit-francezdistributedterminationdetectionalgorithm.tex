%% Generated by Sphinx.
\def\sphinxdocclass{report}
\documentclass[letterpaper,10pt,english]{sphinxmanual}
\ifdefined\pdfpxdimen
   \let\sphinxpxdimen\pdfpxdimen\else\newdimen\sphinxpxdimen
\fi \sphinxpxdimen=.75bp\relax
\ifdefined\pdfimageresolution
    \pdfimageresolution= \numexpr \dimexpr1in\relax/\sphinxpxdimen\relax
\fi
%% let collapsible pdf bookmarks panel have high depth per default
\PassOptionsToPackage{bookmarksdepth=5}{hyperref}

\PassOptionsToPackage{booktabs}{sphinx}
\PassOptionsToPackage{colorrows}{sphinx}

\PassOptionsToPackage{warn}{textcomp}
\usepackage[utf8]{inputenc}
\ifdefined\DeclareUnicodeCharacter
% support both utf8 and utf8x syntaxes
  \ifdefined\DeclareUnicodeCharacterAsOptional
    \def\sphinxDUC#1{\DeclareUnicodeCharacter{"#1}}
  \else
    \let\sphinxDUC\DeclareUnicodeCharacter
  \fi
  \sphinxDUC{00A0}{\nobreakspace}
  \sphinxDUC{2500}{\sphinxunichar{2500}}
  \sphinxDUC{2502}{\sphinxunichar{2502}}
  \sphinxDUC{2514}{\sphinxunichar{2514}}
  \sphinxDUC{251C}{\sphinxunichar{251C}}
  \sphinxDUC{2572}{\textbackslash}
\fi
\usepackage{cmap}
\usepackage[T1]{fontenc}
\usepackage{amsmath,amssymb,amstext}
\usepackage{babel}



\usepackage{tgtermes}
\usepackage{tgheros}
\renewcommand{\ttdefault}{txtt}



\usepackage[Bjarne]{fncychap}
\usepackage{sphinx}

\fvset{fontsize=auto}
\usepackage{geometry}

\usepackage{nbsphinx}

% Include hyperref last.
\usepackage{hyperref}
% Fix anchor placement for figures with captions.
\usepackage{hypcap}% it must be loaded after hyperref.
% Set up styles of URL: it should be placed after hyperref.
\urlstyle{same}

\addto\captionsenglish{\renewcommand{\contentsname}{Contents}}

\usepackage{sphinxmessages}
\setcounter{tocdepth}{1}



\title{Distributed Algorithm on AHCv2: Shavit\sphinxhyphen{}Francez Distributed Termination Detection Algorithm}
\date{May 05, 2024}
\release{V1.0.0}
\author{İmre Kosdik}
\newcommand{\sphinxlogo}{\vbox{}}
\renewcommand{\releasename}{Release}
\makeindex
\begin{document}

\ifdefined\shorthandoff
  \ifnum\catcode`\=\string=\active\shorthandoff{=}\fi
  \ifnum\catcode`\"=\active\shorthandoff{"}\fi
\fi

\pagestyle{empty}
\sphinxmaketitle
\pagestyle{plain}
\sphinxtableofcontents
\pagestyle{normal}
\phantomsection\label{\detokenize{index::doc}}


\sphinxstepscope


\chapter{Shavit\sphinxhyphen{}Francez Termination Detection Algorithm}
\label{\detokenize{docs/ShavitFrancez/ShavitFrancez:shavitfrancezalg}}\label{\detokenize{docs/ShavitFrancez/ShavitFrancez::doc}}
\sphinxstepscope


\section{Abstract}
\label{\detokenize{docs/ShavitFrancez/abstract:abstract}}\label{\detokenize{docs/ShavitFrancez/abstract::doc}}
\sphinxAtStartPar
A computation of a distributed algortihm terminates when the algorithm reaches a state that there are no possible applicable steps. In distributed systems, determining whether a particular computation has terminated is a crucial need because execution of other computations may depend on completion of the computation. Due to the fact that the processes in a distributed system have no knowledge about the global state of the system and do not share any global clock inferring if a distributed computation has ended is a challenging problem. The {\hyperref[\detokenize{docs/ShavitFrancez/algorithm:shavitfrancezterminationdetectionalgorithm}]{\sphinxcrossref{\DUrole{std,std-ref}{Shavit\sphinxhyphen{}Francez Algorithm}}}}, is a fundamental termination detection algorithm that addresses these issues, ensuring that the system reaches a consistent state where all the processes completed their computations before proceeding with other tasks. In this paper, we contribute to the research of termination detection by implementing the {\hyperref[\detokenize{docs/ShavitFrancez/algorithm:shavitfrancezterminationdetectionalgorithm}]{\sphinxcrossref{\DUrole{std,std-ref}{Shavit\sphinxhyphen{}Francez Algorithm}}}} on the AHCv2 Platform, along with anaylzing the message and time complexity by running the algorithm on different network topologies and various number of nodes.

\sphinxstepscope


\section{Introduction}
\label{\detokenize{docs/ShavitFrancez/introduction:introduction}}\label{\detokenize{docs/ShavitFrancez/introduction::doc}}
\sphinxAtStartPar
When a problem requires more than one process to be solved, distributed systems come in handy. These processes work together to solve subproblems. Therefore, it is crucial to identify when a process has completed its execution because its output is used as input to another process to continue its execution. Termination detection in distributed computing is a challenging task in distributed systems because processes are unaware of the global state of the system due to communication delays, and there is no shared global system.

\sphinxAtStartPar
Termination detection algorithms are both interesting and important because they have a significant role in ensuring a consistent state where all processes finished their computations and are ready to proceed with the upcoming tasks. Achieving a consistent state also preserves the correctness of the system. Since resources are shared by many processes in distributed systems, termination detection can also take part in efficient resource management by releasing resources that are no longer needed. Additionally, efficient resource management may  also be useful in preventing deadlocks since the main cause of deadlocks is indefinitely waiting to acquire resources.

\sphinxAtStartPar
Termination is a property of the global state of distributed computing. However, due to the decentralized and asynchronous nature of distributed systems, acquiring the global state of the system is a significant challenge. Naive approaches fail due to the issues related to scalability, concurrency, consistency, and fault tolerance. Since termination detection algorithms rely on additional control messages, the message overhead can greatly impact the performance of the system. Also, as the distributed system expands, the complexity and overhead of maintaining the algorithm can increase resulting in scalability issues. Additionally, designing algorithms such that the underlying computation does not interfere with the ongoing executions is another challenge.

\sphinxAtStartPar
The challenge in detecting termination lies in distributed computing. Previous attempts may have failed due to the complexities arising from concurrency management, achieving scalability, and communication delays. Unlike the previous attempts, The Shavit\sphinxhyphen{}Francez Algorithm is not constrained by predetermined processor setups, doesn’t rely on synchronized communication or basic computation, and doesn’t depend on global information in any process.  Moreover, the {\hyperref[\detokenize{docs/ShavitFrancez/algorithm:shavitfrancezterminationdetectionalgorithm}]{\sphinxcrossref{\DUrole{std,std-ref}{Shavit\sphinxhyphen{}Francez Algorithm}}}} is a worst\sphinxhyphen{}case optimal algorithm.

\sphinxAtStartPar
The {\hyperref[\detokenize{docs/ShavitFrancez/algorithm:shavitfrancezterminationdetectionalgorithm}]{\sphinxcrossref{\DUrole{std,std-ref}{Shavit\sphinxhyphen{}Francez Algorithm}}}} is an effective way to detect termination without interfering with the overall execution of the distributed system. The algorithm doesn’t rely on synchronous communication, simplifying the design and implementation of the system. In contrast to the increasing number of nodes, the message\sphinxhyphen{}sharing overhead from the algorithm remains low, which means it has less impact on the system’s performance. In summary, the {\hyperref[\detokenize{docs/ShavitFrancez/algorithm:shavitfrancezterminationdetectionalgorithm}]{\sphinxcrossref{\DUrole{std,std-ref}{Shavit\sphinxhyphen{}Francez Algorithm}}}} is a foundational method for detecting termination in distributed computing.

\sphinxAtStartPar
Our primary contributions consist of the following:
\begin{itemize}
\item {} 
\sphinxAtStartPar
Implementation of the {\hyperref[\detokenize{docs/ShavitFrancez/algorithm:shavitfrancezterminationdetectionalgorithm}]{\sphinxcrossref{\DUrole{std,std-ref}{Shavit\sphinxhyphen{}Francez Algorithm}}}} on the AHCv2 platform.

\item {} 
\sphinxAtStartPar
Examination of the performance of the algorithm across diverse topologies and usage scenarios.

\end{itemize}

\sphinxstepscope


\section{Shavit\sphinxhyphen{}Francez Termination Detection Algorithm}
\label{\detokenize{docs/ShavitFrancez/algorithm:shavitfrancezalg}}\label{\detokenize{docs/ShavitFrancez/algorithm::doc}}

\subsection{Background and Related Work}
\label{\detokenize{docs/ShavitFrancez/algorithm:background-and-related-work}}
\sphinxAtStartPar
Global termination in a distributed system occurs when all processes reach the local termination state, no messages are in transit, and processes do not send or receive any message. Local termination is the state where the process completed its execution, meaning it is passive (idle) and is ready to continue its computation upon receiving any message. A process is active when it is performing some computation. In a distributed system, only the active processes can send messages. Therefore, a computation in a distributed system terminates when all its processes are idle.

\sphinxAtStartPar
The primary consideration behind the termination detection algorithms is adding a control algorithm to the system running to detect whether the basic algorithm has reached a termination state. The basic algorithm is the one currently running in the distributed system. Initiators of the basic algorithm are active processes and can trigger the execution of the control algorithm. The control algorithm consists of the termination detection and the announcement phases. The messages that the control algorithm sends or receives are control messages. Ideally, the termination detection algorithm should not need additional communication channels to send or receive its control messages, and should not interfere with the basic algorithm running on the system.

\sphinxAtStartPar
{\hyperref[\detokenize{docs/ShavitFrancez/algorithm:shavitfrancezterminationdetectionalgorithm}]{\sphinxcrossref{\DUrole{std,std-ref}{Shavit\sphinxhyphen{}Francez Algorithm}}}} \sphinxcite{docs/ShavitFrancez/algorithm:shavitfrancez1986} is the generalization of Dijkstra\sphinxhyphen{}Scholten Termination Detection Algorithm \sphinxcite{docs/ShavitFrancez/algorithm:dijkstrasholten1980} for distributed systems. In Dijkstra\sphinxhyphen{}Scholten Algorithm, the initiator of the basic algorithm maintains a tree of active processes. If a process makes another process active by sending a message, that process joins the tree as a child of the process. A process can only leave the tree if it transitions to passive state and it has no children in the tree. Once the tree becomes empty, the initiator announces the termination. {\hyperref[\detokenize{docs/ShavitFrancez/algorithm:shavitfrancezterminationdetectionalgorithm}]{\sphinxcrossref{\DUrole{std,std-ref}{Shavit\sphinxhyphen{}Francez Termination Detection Algorithm}}}} \sphinxcite{docs/ShavitFrancez/algorithm:shavitfrancez1986} maintains a forest instead of a single tree due to the nature of the distributed system. Each initiator maintains its tree and constitutes it to the forest. The condition for a process to join a tree is that it is not already a member of any of the trees in the forest. Other than that, the algorithm continues as in the Dijkstra\sphinxhyphen{}Scholten Algorithm. Instead, it starts a wave in which only those processes not part of a tree participate. Because each initiator is only aware of the emptiness of its tree, an empty tree does not guarantee that the whole forest is empty. The wave algorithm ensures that all the trees in the forest collapse before announcing termination. Once the wave decides, the initiator can then announce the termination. A wave algorithm is not complete unless all the processes take part in its execution. Following this property, the algorithm ensures that if none of the waves started by the processes are complete because a process refuses to take part, the initiator maintaining the last tree to be empty will start a wave that eventually decides and announces the termination. \sphinxcite{docs/ShavitFrancez/algorithm:fokking2013}

\sphinxAtStartPar
The wave algorithm we choose for the implementation in {\hyperref[\detokenize{docs/ShavitFrancez/algorithm:shavitfrancezterminationdetectionalgorithm}]{\sphinxcrossref{\DUrole{std,std-ref}{Shavit\sphinxhyphen{}Francez Termination Detection Algorithm}}}}  is {\hyperref[\detokenize{docs/ShavitFrancez/algorithm:echoalgorithm}]{\sphinxcrossref{\DUrole{std,std-ref}{Echo Algorithm}}}} \sphinxcite{docs/ShavitFrancez/algorithm:fokking2013}. The {\hyperref[\detokenize{docs/ShavitFrancez/algorithm:echoalgorithm}]{\sphinxcrossref{\DUrole{std,std-ref}{Echo Algorithm}}}} initiator begins by sending messages to all of its neighbors. If a non\sphinxhyphen{}initiator receives a message for the first time, it sets its parent as the sender process and sends a message to all its neighbors except its parent. After receiving messages from all its neighbors, the non\sphinxhyphen{}initiator notifies its parent. Finally, the initiator receives messages from all its neighbors and decides.


\subsection{Shavit\sphinxhyphen{}Francez Termination Detection Algorithm: Shavit\sphinxhyphen{}Francez Termination Detection Algorithm}
\label{\detokenize{docs/ShavitFrancez/algorithm:shavit-francez-termination-detection-algorithm-shavitfrancezalg}}
\sphinxAtStartPar
The {\hyperref[\detokenize{docs/ShavitFrancez/algorithm:shavitfrancezterminationdetectionalgorithm}]{\sphinxcrossref{\DUrole{std,std-ref}{Shavit\sphinxhyphen{}Francez Termination Detection Algorithm}}}} is proposed by Nir Shavit and Nissim Francez to detect termination in the distributed system algorithms. General flow of execution of the algorithm is as follows:
\begin{enumerate}
\sphinxsetlistlabels{\arabic}{enumi}{enumii}{}{.}%
\item {} 
\sphinxAtStartPar
If a process is the initiator of the basic algorithm, it sets its active property to true, indicating that it is doing some computation at the time. (Line 5)

\item {} 
\sphinxAtStartPar
If a process sends a basic message while executing the basic algorithm, then it increases its number of children by 1, because the process that is sent message becomes its children. (Line 8)

\item {} 
\sphinxAtStartPar
If a process receives a basic message while it is passive, then it becomes active (Line 12) and sets its parent to the process sending the basic message (Line 13). If it was already active, then it informs the sending process with an ACKNOWLEDGE message. (Line 15)

\item {} 
\sphinxAtStartPar
If a process receives an ACKNOWLEDGE message, it decreases its number of children by 1 (Line 19), calls the LeaveTree procedure. (Line 20)

\item {} 
\sphinxAtStartPar
If at some point a process becomes passive then it sets its active property to false (Line 23), calls the LeaveTree procedure. (Line 24)

\item {} 
\sphinxAtStartPar
Inside the LeaveTree procedure, a passive process with no children sends its parent an ACKNOWLEDGE message (Line 29), and then sets its parent to None (Line 30).

\item {} 
\sphinxAtStartPar
If a passive process with no children has no parent at all, it starts a wave inside the LeaveTree procedure. (Line 32).

\item {} 
\sphinxAtStartPar
A a passive process with no children receiving a wave message acts as what the wave algorithm dictates (Line 37) and if the wave algorithm decides, it announces termination. (Line 38)

\end{enumerate}
\sphinxSetupCaptionForVerbatim{Shavit\sphinxhyphen{}Francez Termination Detection Algorithm}
\def\sphinxLiteralBlockLabel{\label{\detokenize{docs/ShavitFrancez/algorithm:id12}}\label{\detokenize{docs/ShavitFrancez/algorithm:shavitfrancezterminationdetectionalgorithm}}}
\begin{sphinxVerbatim}[commandchars=\\\{\},numbers=left,firstnumber=1,stepnumber=1]
1   bool active\PYGZlt{}p\PYGZgt{} // set when p becomes active, and reset when p becomes passive
2   nat cc\PYGZlt{}p\PYGZgt{} // keeps track of the number of children of p in its tree
3   proc parent\PYGZlt{}p\PYGZgt{} // the parent of p in a tree in the forest

4   if p is an initiator then
5       active\PYGZlt{}p\PYGZgt{} \PYGZlt{}\PYGZhy{} true
6   end if

7   if p sends a basic message then
8       cc\PYGZlt{}p\PYGZgt{} \PYGZlt{}\PYGZhy{} cc\PYGZlt{}p\PYGZgt{} + 1
9   end if

10  if p receives a basic message from a neighbor q then
11    if active\PYGZlt{}p\PYGZgt{} = false then
12        active\PYGZlt{}p\PYGZgt{} \PYGZlt{}\PYGZhy{} true
13        parent\PYGZlt{}p\PYGZgt{} \PYGZlt{}\PYGZhy{} q
14    else
15        send \PYGZlt{}ack\PYGZgt{} to q
16    end if
17  end if

18  if p receives \PYGZlt{}ack\PYGZgt{}
19      cc\PYGZlt{}p\PYGZgt{} \PYGZlt{}\PYGZhy{} cc\PYGZlt{}p\PYGZgt{} \PYGZhy{} 1
20      perform procedure LeaveTree\PYGZlt{}p\PYGZgt{}
21  end if

22  if p becomes passive
23      active\PYGZlt{}p\PYGZgt{} \PYGZlt{}\PYGZhy{} false
24      perform procedure LeaveTree\PYGZlt{}p\PYGZgt{};
25  end if

26  Procedure LeaveTree\PYGZlt{}p\PYGZgt{}
27      if active\PYGZlt{}p\PYGZgt{} = false and cc\PYGZlt{}p\PYGZgt{} = 0 then
28          if parent\PYGZlt{}p\PYGZgt{} != ┴ then
29              send \PYGZlt{}ack\PYGZgt{} to parent\PYGZlt{}p\PYGZgt{}
30              parent\PYGZlt{}p\PYGZgt{} \PYGZlt{}\PYGZhy{} ┴
31          else
32              start a wave, tagged with p
33          end if
34      end if

35  if p receives a wave message then
36      if active\PYGZlt{}p\PYGZgt{} = false and cc\PYGZlt{}p\PYGZgt{} = 0 then
37          act according to the wave algorithm
38          in the case of a decive event, call Announce
39      end if
40  end if
\end{sphinxVerbatim}


\subsection{Echo Algorithm:}
\label{\detokenize{docs/ShavitFrancez/algorithm:echo-algorithm}}
\sphinxAtStartPar
The {\hyperref[\detokenize{docs/ShavitFrancez/algorithm:echoalgorithm}]{\sphinxcrossref{\DUrole{std,std-ref}{Echo Algorithm}}}} \sphinxcite{docs/ShavitFrancez/algorithm:fokking2013} takes part in making sure that all the trees in the forest collapsed and thus, concluding that the basic algorithm terminated. Since this paper focuses on the implementation details of the {\hyperref[\detokenize{docs/ShavitFrancez/algorithm:shavitfrancezterminationdetectionalgorithm}]{\sphinxcrossref{\DUrole{std,std-ref}{Shavit\sphinxhyphen{}Francez Termination Detection Algorithm}}}}, we do not explicitly describe the pseudocode we provided for the wave algorithm. We only add the pseudocode here because we make use of this algorithm while implementing the {\hyperref[\detokenize{docs/ShavitFrancez/algorithm:shavitfrancezterminationdetectionalgorithm}]{\sphinxcrossref{\DUrole{std,std-ref}{Shavit\sphinxhyphen{}Francez Termination Detection Algorithm}}}}.
\sphinxSetupCaptionForVerbatim{Echo Algorithm}
\def\sphinxLiteralBlockLabel{\label{\detokenize{docs/ShavitFrancez/algorithm:id13}}\label{\detokenize{docs/ShavitFrancez/algorithm:echoalgorithm}}}
\begin{sphinxVerbatim}[commandchars=\\\{\},numbers=left,firstnumber=1,stepnumber=1]
1   nat received\PYGZlt{}p\PYGZgt{};
2   proc parent\PYGZlt{}p\PYGZgt{};

3   if p is the initiator then
4       send \PYGZlt{}wave\PYGZgt{} to each r in Neighbors\PYGZlt{}p\PYGZgt{}
5   end if

6   if p receives a \PYGZlt{}wave\PYGZgt{} from neighbor q then
7       received\PYGZlt{}p\PYGZgt{} \PYGZlt{}\PYGZhy{} received\PYGZlt{}p\PYGZgt{} + 1
8       if parent\PYGZlt{}p\PYGZgt{} != ┴ and p is a non\PYGZhy{}initiator then
9           parent\PYGZlt{}p\PYGZgt{} \PYGZlt{}\PYGZhy{} q
10          if |Neighbors\PYGZlt{}p\PYGZgt{}| \PYGZgt{} 1 then
11              send \PYGZlt{}wave\PYGZgt{} to each r in Neighbors\PYGZlt{}p\PYGZgt{}\PYGZbs{}\PYGZob{}q\PYGZcb{}
12          else
13              send \PYGZlt{}wave\PYGZgt{} to q
14          end if
15      else if received\PYGZlt{}p\PYGZgt{} = |Neighbors\PYGZlt{}p\PYGZgt{}| then
16          if parent\PYGZlt{}p\PYGZgt{} != ┴ then
17              send \PYGZlt{}wave\PYGZgt{} to parent\PYGZlt{}p\PYGZgt{}
18          else
19              decide
20          end if
21      end if
22  end if
\end{sphinxVerbatim}


\subsection{Example With Terminating Distributed System Algorithm}
\label{\detokenize{docs/ShavitFrancez/algorithm:example-with-terminating-distributed-system-algorithm}}

\begin{savenotes}\sphinxattablestart
\sphinxthistablewithglobalstyle
\centering
\begin{tabulary}{\linewidth}[t]{TT}
\sphinxtoprule
\sphinxtableatstartofbodyhook\begin{sphinxfigure-in-table}
\centering
\capstart
\noindent\sphinxincludegraphics[width=766\sphinxpxdimen,height=558\sphinxpxdimen]{{shavit_step1}.png}
\sphinxfigcaption{Step 1}\label{\detokenize{docs/ShavitFrancez/algorithm:id14}}\end{sphinxfigure-in-table}\relax
&\begin{sphinxfigure-in-table}
\centering
\capstart
\noindent\sphinxincludegraphics[width=766\sphinxpxdimen,height=558\sphinxpxdimen]{{shavit_step2}.png}
\sphinxfigcaption{Step 2}\label{\detokenize{docs/ShavitFrancez/algorithm:id15}}\end{sphinxfigure-in-table}\relax
\\
\sphinxhline\begin{sphinxfigure-in-table}
\centering
\capstart
\noindent\sphinxincludegraphics[width=766\sphinxpxdimen,height=558\sphinxpxdimen]{{shavit_step3}.png}
\sphinxfigcaption{Step 3}\label{\detokenize{docs/ShavitFrancez/algorithm:id16}}\end{sphinxfigure-in-table}\relax
&\begin{sphinxfigure-in-table}
\centering
\capstart
\noindent\sphinxincludegraphics[width=766\sphinxpxdimen,height=558\sphinxpxdimen]{{shavit_step4}.png}
\sphinxfigcaption{Step 4}\label{\detokenize{docs/ShavitFrancez/algorithm:id17}}\end{sphinxfigure-in-table}\relax
\\
\sphinxhline\begin{sphinxfigure-in-table}
\centering
\capstart
\noindent\sphinxincludegraphics[width=766\sphinxpxdimen,height=558\sphinxpxdimen]{{shavit_step5}.png}
\sphinxfigcaption{Step 5}\label{\detokenize{docs/ShavitFrancez/algorithm:id18}}\end{sphinxfigure-in-table}\relax
&\begin{sphinxfigure-in-table}
\centering
\capstart
\noindent\sphinxincludegraphics[width=766\sphinxpxdimen,height=558\sphinxpxdimen]{{shavit_step6}.png}
\sphinxfigcaption{Step 6}\label{\detokenize{docs/ShavitFrancez/algorithm:id19}}\end{sphinxfigure-in-table}\relax
\\
\sphinxbottomrule
\end{tabulary}
\sphinxtableafterendhook\par
\sphinxattableend\end{savenotes}

\sphinxAtStartPar
Assume that there are three processes p, q, r in an undirected network. One way to execute the {\hyperref[\detokenize{docs/ShavitFrancez/algorithm:shavitfrancezterminationdetectionalgorithm}]{\sphinxcrossref{\DUrole{std,std-ref}{Shavit\sphinxhyphen{}Francez Algorithm}}}} is as follows:
\begin{enumerate}
\sphinxsetlistlabels{\arabic}{enumi}{enumii}{}{.}%
\item {} 
\sphinxAtStartPar
At the start, the initiators p and q both send a basic message to r, and set cc\textless{}p\textgreater{} and cc\textless{}q\textgreater{} to 1. Next, p and q become passive.(See Figure 1)

\item {} 
\sphinxAtStartPar
Upon receipt of the basic message from p, r becomes active and makes p its parent. Next, r receives the basic message from q, and sends back an acknowledgment, which causes q to decrease cc\textless{}q\textgreater{} to 0.(See Figure 2)

\item {} 
\sphinxAtStartPar
Since q became passive as the root of a tree, and cc\textless{}q\textgreater{} = 0, it starts a wave. This wave does not complete, because p and r refuse to participate.(See Figure 3)

\item {} 
\sphinxAtStartPar
r sends a basic message to q, and sets cc\textless{}r\textgreater{} to 1. Next, r becomes passive.(See Figure 4)

\item {} 
\sphinxAtStartPar
Upon receipt of the basic message from r, q becomes active, and makes r its parent. Next, q becomes passive, and sends an acknowledgment to its parent r, which causes r to decrease cc\textless{}r\textgreater{} to 0. Since r is passive and cc\textless{}r\textgreater{} = 0, it sends an acknowledgment to its parent p, which causes p to decrease cc\textless{}p\textgreater{} to 0.(See Figure 5)

\item {} 
\sphinxAtStartPar
Since p became passive as the root of a tree, and cc\textless{}p\textgreater{} = 0, it starts a wave. This wave completes, so that p calls Announce.(See Figure 6)

\end{enumerate}


\subsection{Example With Non\sphinxhyphen{}Terminating Distributed System Algorithm}
\label{\detokenize{docs/ShavitFrancez/algorithm:example-with-non-terminating-distributed-system-algorithm}}

\begin{savenotes}\sphinxattablestart
\sphinxthistablewithglobalstyle
\centering
\begin{tabulary}{\linewidth}[t]{T}
\sphinxtoprule
\sphinxtableatstartofbodyhook\begin{sphinxfigure-in-table}
\centering
\capstart
\noindent\sphinxincludegraphics[width=957\sphinxpxdimen,height=481\sphinxpxdimen]{{shavit_ex2_step1}.png}
\sphinxfigcaption{Step 1}\label{\detokenize{docs/ShavitFrancez/algorithm:id20}}\end{sphinxfigure-in-table}\relax
\\
\sphinxhline\begin{sphinxfigure-in-table}
\centering
\capstart
\noindent\sphinxincludegraphics[width=957\sphinxpxdimen,height=481\sphinxpxdimen]{{shavit_ex2_step2}.png}
\sphinxfigcaption{Step 2}\label{\detokenize{docs/ShavitFrancez/algorithm:id21}}\end{sphinxfigure-in-table}\relax
\\
\sphinxhline\begin{sphinxfigure-in-table}
\centering
\capstart
\noindent\sphinxincludegraphics[width=957\sphinxpxdimen,height=481\sphinxpxdimen]{{shavit_ex2_step3}.png}
\sphinxfigcaption{Step 3}\label{\detokenize{docs/ShavitFrancez/algorithm:id22}}\end{sphinxfigure-in-table}\relax
\\
\sphinxbottomrule
\end{tabulary}
\sphinxtableafterendhook\par
\sphinxattableend\end{savenotes}

\sphinxAtStartPar
Assume that there are three processes p, q, r in an undirected network. One way to execute the {\hyperref[\detokenize{docs/ShavitFrancez/algorithm:shavitfrancezterminationdetectionalgorithm}]{\sphinxcrossref{\DUrole{std,std-ref}{Shavit\sphinxhyphen{}Francez Algorithm}}}} is as follows:
\begin{enumerate}
\sphinxsetlistlabels{\arabic}{enumi}{enumii}{}{.}%
\item {} 
\sphinxAtStartPar
At the start, the initiators p and r both send a basic message to q, and set cc\textless{}p\textgreater{} and cc\textless{}r\textgreater{} to 1. (See Figure 7)

\item {} 
\sphinxAtStartPar
Upon receipt of the basic message from p, q becomes active and makes p its parent. Next, q receives the basic message from r, and sends back an acknowledgment, which causes r to decrease cc\textless{}r\textgreater{} to 0. (See Figure 2)

\item {} 
\sphinxAtStartPar
Next, r becomes passive.

\item {} 
\sphinxAtStartPar
Since r became passive as the root of a tree, and cc\textless{}r\textgreater{} = 0, it starts a wave. This wave does not complete, because p and q refuse to participate.(See Figure 3)

\item {} 
\sphinxAtStartPar
Since neither p nor q becomes passive at some point, the algorithm cannot complete the wave and cannot announce termination.

\end{enumerate}


\subsection{Correctness}
\label{\detokenize{docs/ShavitFrancez/algorithm:correctness}}\begin{enumerate}
\sphinxsetlistlabels{\arabic}{enumi}{enumii}{}{.}%
\item {} 
\sphinxAtStartPar
\sphinxstylestrong{Safety}: The \sphinxstyleemphasis{Announce} is called when a decision occurs in the wave algorithm. This implies that each process p has sent a wave message or has decided, and the algorithm implies that cc\textless{}p\textgreater{} was 0 when p did so. No action makes cc\textless{}p\textgreater{} more than 0 again, so (for each p) cc\textless{}p\textgreater{} is 0 when \sphinxstyleemphasis{Announce} is called. \sphinxcite{docs/ShavitFrancez/algorithm:tel2001}

\item {} 
\sphinxAtStartPar
\sphinxstylestrong{Liveness}: Assume that the basic computation has terminated. Within a finite number of steps the termination\sphinxhyphen{}detection algorithm reaches a terminal configuration, and as in the correctness statement below it can be shown that in this configuration the forest is empty. Consequently, all events of the wave are enabled in every process, and that the configuration is terminal now implies that all events of the wave have been executed, including at least one decision, which caused a call to \sphinxstyleemphasis{Announce}. \sphinxcite{docs/ShavitFrancez/algorithm:tel2001}

\item {} 
\sphinxAtStartPar
\sphinxstylestrong{Correctness}: Define S to be the sum of all cc\textless{}p\textgreater{} for each process p. Initially S is zero, S is incremented when a basic message is sent, S is decremented when a control message is received, and S is never negative. This implies that the number of control messages never exceeds the number of basic messages in any computation. \sphinxcite{docs/ShavitFrancez/algorithm:tel2001}

\end{enumerate}


\subsection{Complexity}
\label{\detokenize{docs/ShavitFrancez/algorithm:complexity}}\begin{enumerate}
\sphinxsetlistlabels{\arabic}{enumi}{enumii}{}{.}%
\item {} 
\sphinxAtStartPar
{\hyperref[\detokenize{docs/ShavitFrancez/algorithm:shavitfrancezterminationdetectionalgorithm}]{\sphinxcrossref{\DUrole{std,std-ref}{Shavit\sphinxhyphen{}Francez Algorithm}}}}: The worst case message complexity  is O(M + W) where M is the number of the messages sent by the underlying computation and W is a message exchange complexity of the wave algorithm, which is 2E for {\hyperref[\detokenize{docs/ShavitFrancez/algorithm:echoalgorithm}]{\sphinxcrossref{\DUrole{std,std-ref}{Echo Algorithm}}}} where E is. \sphinxcite{docs/ShavitFrancez/algorithm:tel2001}

\item {} 
\sphinxAtStartPar
{\hyperref[\detokenize{docs/ShavitFrancez/algorithm:echoalgorithm}]{\sphinxcrossref{\DUrole{std,std-ref}{Echo Algorithm}}}}: The message complexity is O(2E), where E is the number of edges. \sphinxcite{docs/ShavitFrancez/algorithm:fokking2013}

\end{enumerate}


\subsection{References}
\label{\detokenize{docs/ShavitFrancez/algorithm:references}}
\sphinxstepscope


\section{Implementation, Results and Discussion}
\label{\detokenize{docs/ShavitFrancez/results:implementation-results-and-discussion}}\label{\detokenize{docs/ShavitFrancez/results::doc}}

\subsection{Implementation and Methodology}
\label{\detokenize{docs/ShavitFrancez/results:implementation-and-methodology}}
\sphinxAtStartPar
We utilized the Python (version 3.12) scripting language and the Ad\sphinxhyphen{}Hoc Computing (adhoccomputing) library while implementing the Shavit\sphinxhyphen{}Francez Termination Detection Algorithm. We also employed the networkx library to generate various network topologies and the matplotlib library to visualize them. Each component in the topology can be the initiator for the termination detection algorithm. It is up to us which component to choose the initiator/initiators. After that, we must send an event to the initiators to execute the termination detection algorithm. Either initiator components can send the event to themselves, or other non\sphinxhyphen{}initator nodes can send it to the initiators. Since termination detection is the algorithm that runs on top of the basic algorithm running in the system, we needed to simulate a basic algorithm by creating additional messages that we could send to the component externally. We use “\sphinxstyleemphasis{BECOMEPASSIVE}” message to simulate processes finishing their execution and “\sphinxstyleemphasis{SENDBASICMESSAGE}” to simulate messages that the basic algorithm exchanges on its execution. An important consideration is that one can only send these messages if the process is active. Another consideration is that, the components need to be aware of who is executing the control algorithm. Therefore, the process starting the algorithm send a message to its neighbors indicating that it is the initiator for this execution.

\sphinxAtStartPar
For a distributed system in that its processes never become passive, we should expect that the algorithm does not announce the termination and, therefore, no output in the command prompt. As an example, we can consider a system with deadlocks. Since none of the processes can continue because they need resources from others, the algorithm cannot announce the termination. To create this scenario, we could think that the “SENDBASICMESSAGE” event acts as a “REQUEST” and create a cyclic graph. In other cases, sending a “BECOMEPASSIVE” event to a process acts as if the process finishing its execution, and we should see that the algorithm announces the termination in the commant prompt.

\sphinxAtStartPar
We implemented both the Echo Algorithm and the Shavit\sphinxhyphen{}Francez Termination Detection Algorithm by employing the pseudocode descriptions given in \sphinxcite{docs/ShavitFrancez/algorithm:fokking2013}. We used the same message types given in the descriptions to achieve the message passing between the components. The make the component who is the initiator of the basic algorithm send itself “DETECTTERMINATION” message to trigger the algorithm. After that, depending on the basic\sphinxhyphen{}messages exchanged between the processes and the status of the processes, the algorithm announces the termination.


\subsection{Results}
\label{\detokenize{docs/ShavitFrancez/results:results}}

\subsection{Discussion}
\label{\detokenize{docs/ShavitFrancez/results:discussion}}
\sphinxstepscope


\section{Conclusion}
\label{\detokenize{docs/ShavitFrancez/conclusion:conclusion}}\label{\detokenize{docs/ShavitFrancez/conclusion::doc}}
\sphinxAtStartPar
In general a short summarizing paragraph will do, and under no circumstances should the paragraph simply repeat material from the Abstract or Introduction. In some cases it’s possible to now make the original claims more concrete, e.g., by referring to quantitative performance results {[}Widom2006{]}.

\sphinxAtStartPar
The conclusion is where you build upon your discussion and try to refer your findings to other research and to the world at large. In a short research paper, it may be a paragraph or two, or practically non\sphinxhyphen{}existent. In a dissertation, it may well be the most important part of the entire paper \sphinxhyphen{} not only does it describe the results and discussion in detail, it emphasizes the importance of the results in the field, and ties it in with the previous research. Some research papers require a recommendations section, postulating that further directions of the research, as well as highlighting how any flaws affected the results. In this case, you should suggest any improvements that could be made to the research design {[}Shuttleworth2016{]}.

\sphinxstepscope


\chapter{Assessment Rubric}
\label{\detokenize{docs/rubric:assessment-rubric}}\label{\detokenize{docs/rubric::doc}}
\sphinxAtStartPar
Your work and documentation will be assessed based on the following list of criteria.


\section{Organization and Style}
\label{\detokenize{docs/rubric:organization-and-style}}
\sphinxAtStartPar
{[}15 points{]} The documentation states  title, author names, affiliations and date. The format follows this style?
\begin{enumerate}
\sphinxsetlistlabels{\arabic}{enumi}{enumii}{}{.}%
\item {} 
\sphinxAtStartPar
Structure and Organization: Does the organization of the paper enhance understanding of the material? Is the flow logical with appropriate transitions between sections?

\item {} 
\sphinxAtStartPar
Technical Exposition: Is the technical material presented clearly and logically? Is the material presented at the appropriate level of detail?

\item {} 
\sphinxAtStartPar
Clarity: Is the writing clear, unambiguous and direct? Is there excessive use of jargon, acronyms or undefined terms?

\item {} 
\sphinxAtStartPar
Style: Does the writing adhere to conventional rules of grammar and style? Are the references sufficient and appropriate?

\item {} 
\sphinxAtStartPar
Length: Is the length of the paper appropriate to the technical content?

\item {} 
\sphinxAtStartPar
Illustrations: Do the figures and tables enhance understanding of the text? Are they well explained? Are they of appropriate number, format and size?

\item {} 
\sphinxAtStartPar
Documentation style: Did you follow the expected documentation style (rst or latex)?

\end{enumerate}


\section{Abstract}
\label{\detokenize{docs/rubric:abstract}}
\sphinxAtStartPar
{[}10 points{]} Does the abstract summarize the documentation?
\begin{enumerate}
\sphinxsetlistlabels{\arabic}{enumi}{enumii}{}{.}%
\item {} 
\sphinxAtStartPar
Motivation/problem statement: Why do we care about the problem? What practical, scientific or theoretical gap is your research filling?

\item {} 
\sphinxAtStartPar
Methods/procedure/approach: What did you actually do to get your results?

\item {} 
\sphinxAtStartPar
Results/findings/product: As a result of completing the above procedure, what did you learn/invent/create? What are the main learning points?

\item {} 
\sphinxAtStartPar
Conclusion/implications: What are the larger implications  of your findings, especially for the problem/gap identified?

\end{enumerate}


\section{Introduction and the Problem}
\label{\detokenize{docs/rubric:introduction-and-the-problem}}
\sphinxAtStartPar
{[}15 points{]} The problem section must be specific. The title of the section must indicate your problem. Do not use generic titles.
\begin{enumerate}
\sphinxsetlistlabels{\arabic}{enumi}{enumii}{}{.}%
\item {} 
\sphinxAtStartPar
Is the problem clearly stated?

\item {} 
\sphinxAtStartPar
Is the problem practically important?

\item {} 
\sphinxAtStartPar
What is the purpose of the study?

\item {} 
\sphinxAtStartPar
What is the hypothesis?

\item {} 
\sphinxAtStartPar
Are the key terms defined?

\end{enumerate}


\section{Background and Related Work}
\label{\detokenize{docs/rubric:background-and-related-work}}
\sphinxAtStartPar
{[}15 points{]} Does the documentation present the background and related work in separate sections.
\begin{enumerate}
\sphinxsetlistlabels{\arabic}{enumi}{enumii}{}{.}%
\item {} 
\sphinxAtStartPar
Are the cited sources pertinent to the study?

\item {} 
\sphinxAtStartPar
Is the review too broad or too narrow?

\item {} 
\sphinxAtStartPar
Are the references/citation recent or appropriate?

\item {} 
\sphinxAtStartPar
Is there any evidence of bias?

\end{enumerate}


\section{Implementation and Methodology}
\label{\detokenize{docs/rubric:implementation-and-methodology}}
\sphinxAtStartPar
{[}15 points{]} Does the documentation present the design of the study.
\begin{enumerate}
\sphinxsetlistlabels{\arabic}{enumi}{enumii}{}{.}%
\item {} 
\sphinxAtStartPar
What research methodology was used?

\item {} 
\sphinxAtStartPar
Was it a replica study or an original study?

\item {} 
\sphinxAtStartPar
What measurement tools were used?

\item {} 
\sphinxAtStartPar
How were the procedures structured and the implementation done?

\item {} 
\sphinxAtStartPar
Were extensive exprimentations conducted providing not only means but also confidence intervals?

\item {} 
\sphinxAtStartPar
What are the assessed parameters and were they adequate?

\item {} 
\sphinxAtStartPar
How was sampling and measurement performed?

\end{enumerate}


\section{Analysis and Discussion}
\label{\detokenize{docs/rubric:analysis-and-discussion}}
\sphinxAtStartPar
{[}15 points{]} Does the documentation present the analysis?
\begin{enumerate}
\sphinxsetlistlabels{\arabic}{enumi}{enumii}{}{.}%
\item {} 
\sphinxAtStartPar
Did you collected enough and adequate data for analysis?

\item {} 
\sphinxAtStartPar
How was data analyzed?

\item {} 
\sphinxAtStartPar
Was data qualitative or quantitative?

\item {} 
\sphinxAtStartPar
Did you provide main learning points based on analysis and results?

\item {} 
\sphinxAtStartPar
Did findings support the hypothesis and purpose?

\item {} 
\sphinxAtStartPar
Did you provide discussion as to the main learning points?

\item {} 
\sphinxAtStartPar
Were weaknesses and problems discussed?

\end{enumerate}


\section{Conclusion and Future Work}
\label{\detokenize{docs/rubric:conclusion-and-future-work}}
\sphinxAtStartPar
{[}15 points{]} Does the documentation state the conclusion and future work clearly?
\begin{enumerate}
\sphinxsetlistlabels{\arabic}{enumi}{enumii}{}{.}%
\item {} 
\sphinxAtStartPar
Are the conclusions of the study related to the original purpose?

\item {} 
\sphinxAtStartPar
Were the implications discussed?

\item {} 
\sphinxAtStartPar
Whom will the results and conclusions effect?

\item {} 
\sphinxAtStartPar
What recommendations were made at the conclusion?

\item {} 
\sphinxAtStartPar
Did you provide future work and suggestions?

\end{enumerate}

\sphinxstepscope


\chapter{Code Documentation}
\label{\detokenize{docs/ShavitFrancez/code:code-documentation}}\label{\detokenize{docs/ShavitFrancez/code::doc}}

\begin{savenotes}\sphinxattablestart
\sphinxthistablewithglobalstyle
\sphinxthistablewithnovlinesstyle
\centering
\begin{tabulary}{\linewidth}[t]{\X{1}{2}\X{1}{2}}
\sphinxtoprule
\sphinxtableatstartofbodyhook
\sphinxAtStartPar
{\hyperref[\detokenize{docs/ShavitFrancez/generated/ShavitFrancez.ShavitFrancez:module-ShavitFrancez.ShavitFrancez}]{\sphinxcrossref{\sphinxcode{\sphinxupquote{ShavitFrancez.ShavitFrancez}}}}}
&
\sphinxAtStartPar

\\
\sphinxbottomrule
\end{tabulary}
\sphinxtableafterendhook\par
\sphinxattableend\end{savenotes}

\sphinxstepscope


\section{ShavitFrancez.ShavitFrancez}
\label{\detokenize{docs/ShavitFrancez/generated/ShavitFrancez.ShavitFrancez:module-ShavitFrancez.ShavitFrancez}}\label{\detokenize{docs/ShavitFrancez/generated/ShavitFrancez.ShavitFrancez:shavitfrancez-shavitfrancez}}\label{\detokenize{docs/ShavitFrancez/generated/ShavitFrancez.ShavitFrancez::doc}}\index{module@\spxentry{module}!ShavitFrancez.ShavitFrancez@\spxentry{ShavitFrancez.ShavitFrancez}}\index{ShavitFrancez.ShavitFrancez@\spxentry{ShavitFrancez.ShavitFrancez}!module@\spxentry{module}}\subsubsection*{Classes}


\begin{savenotes}\sphinxattablestart
\sphinxthistablewithglobalstyle
\sphinxthistablewithnovlinesstyle
\centering
\begin{tabulary}{\linewidth}[t]{\X{1}{2}\X{1}{2}}
\sphinxtoprule
\sphinxtableatstartofbodyhook\sphinxbottomrule
\end{tabulary}
\sphinxtableafterendhook\par
\sphinxattableend\end{savenotes}
\index{ShavitFrancezComponentModel (class in ShavitFrancez.ShavitFrancez)@\spxentry{ShavitFrancezComponentModel}\spxextra{class in ShavitFrancez.ShavitFrancez}}

\begin{fulllineitems}
\phantomsection\label{\detokenize{docs/ShavitFrancez/generated/ShavitFrancez.ShavitFrancez:ShavitFrancez.ShavitFrancez.ShavitFrancezComponentModel}}
\pysigstartsignatures
\pysiglinewithargsret{\sphinxbfcode{\sphinxupquote{class\DUrole{w}{ }}}\sphinxcode{\sphinxupquote{ShavitFrancez.ShavitFrancez.}}\sphinxbfcode{\sphinxupquote{ShavitFrancezComponentModel}}}{\sphinxparam{\DUrole{n}{componentname}}\sphinxparamcomma \sphinxparam{\DUrole{n}{componentinstancenumber}}\sphinxparamcomma \sphinxparam{\DUrole{n}{topology}}\sphinxparamcomma \sphinxparam{\DUrole{n}{context}\DUrole{o}{=}\DUrole{default_value}{None}}\sphinxparamcomma \sphinxparam{\DUrole{n}{configurationparameters}\DUrole{o}{=}\DUrole{default_value}{None}}\sphinxparamcomma \sphinxparam{\DUrole{n}{num\_worker\_threads}\DUrole{o}{=}\DUrole{default_value}{1}}\sphinxparamcomma \sphinxparam{\DUrole{n}{child\_conn}\DUrole{o}{=}\DUrole{default_value}{None}}\sphinxparamcomma \sphinxparam{\DUrole{n}{node\_queues}\DUrole{o}{=}\DUrole{default_value}{None}}\sphinxparamcomma \sphinxparam{\DUrole{n}{channel\_queues}\DUrole{o}{=}\DUrole{default_value}{None}}}{}
\pysigstopsignatures\index{on\_init() (ShavitFrancez.ShavitFrancez.ShavitFrancezComponentModel method)@\spxentry{on\_init()}\spxextra{ShavitFrancez.ShavitFrancez.ShavitFrancezComponentModel method}}

\begin{fulllineitems}
\phantomsection\label{\detokenize{docs/ShavitFrancez/generated/ShavitFrancez.ShavitFrancez:ShavitFrancez.ShavitFrancez.ShavitFrancezComponentModel.on_init}}
\pysigstartsignatures
\pysiglinewithargsret{\sphinxbfcode{\sphinxupquote{on\_init}}}{\sphinxparam{\DUrole{n}{eventobj}}}{}
\pysigstopsignatures
\end{fulllineitems}

\index{on\_receiving\_send\_basic\_message() (ShavitFrancez.ShavitFrancez.ShavitFrancezComponentModel method)@\spxentry{on\_receiving\_send\_basic\_message()}\spxextra{ShavitFrancez.ShavitFrancez.ShavitFrancezComponentModel method}}

\begin{fulllineitems}
\phantomsection\label{\detokenize{docs/ShavitFrancez/generated/ShavitFrancez.ShavitFrancez:ShavitFrancez.ShavitFrancez.ShavitFrancezComponentModel.on_receiving_send_basic_message}}
\pysigstartsignatures
\pysiglinewithargsret{\sphinxbfcode{\sphinxupquote{on\_receiving\_send\_basic\_message}}}{\sphinxparam{\DUrole{n}{eventobj}}}{}
\pysigstopsignatures
\end{fulllineitems}

\index{on\_receiving\_detect\_termination() (ShavitFrancez.ShavitFrancez.ShavitFrancezComponentModel method)@\spxentry{on\_receiving\_detect\_termination()}\spxextra{ShavitFrancez.ShavitFrancez.ShavitFrancezComponentModel method}}

\begin{fulllineitems}
\phantomsection\label{\detokenize{docs/ShavitFrancez/generated/ShavitFrancez.ShavitFrancez:ShavitFrancez.ShavitFrancez.ShavitFrancezComponentModel.on_receiving_detect_termination}}
\pysigstartsignatures
\pysiglinewithargsret{\sphinxbfcode{\sphinxupquote{on\_receiving\_detect\_termination}}}{\sphinxparam{\DUrole{n}{eventobj}}}{}
\pysigstopsignatures
\sphinxAtStartPar
This method makes the proces receiving the DETECTTERMINATION event
active and sets its parent to itself. Then it makes the process send
basic messages to all of its neighbors.

\end{fulllineitems}

\index{on\_receiving\_basic\_message() (ShavitFrancez.ShavitFrancez.ShavitFrancezComponentModel method)@\spxentry{on\_receiving\_basic\_message()}\spxextra{ShavitFrancez.ShavitFrancez.ShavitFrancezComponentModel method}}

\begin{fulllineitems}
\phantomsection\label{\detokenize{docs/ShavitFrancez/generated/ShavitFrancez.ShavitFrancez:ShavitFrancez.ShavitFrancez.ShavitFrancezComponentModel.on_receiving_basic_message}}
\pysigstartsignatures
\pysiglinewithargsret{\sphinxbfcode{\sphinxupquote{on\_receiving\_basic\_message}}}{\sphinxparam{\DUrole{n}{eventobj}}}{}
\pysigstopsignatures
\sphinxAtStartPar
This method makes the process receiving the basic message active if not already and sets its parent
to the process sending the basic message. If the process was already active, then it sends an
acknowledge message to the process sending the basic message.

\end{fulllineitems}

\index{on\_receiving\_acknowledge\_message() (ShavitFrancez.ShavitFrancez.ShavitFrancezComponentModel method)@\spxentry{on\_receiving\_acknowledge\_message()}\spxextra{ShavitFrancez.ShavitFrancez.ShavitFrancezComponentModel method}}

\begin{fulllineitems}
\phantomsection\label{\detokenize{docs/ShavitFrancez/generated/ShavitFrancez.ShavitFrancez:ShavitFrancez.ShavitFrancez.ShavitFrancezComponentModel.on_receiving_acknowledge_message}}
\pysigstartsignatures
\pysiglinewithargsret{\sphinxbfcode{\sphinxupquote{on\_receiving\_acknowledge\_message}}}{\sphinxparam{\DUrole{n}{eventobj}}}{}
\pysigstopsignatures
\sphinxAtStartPar
This method decreases the number of children of the process receiving
the acknowledge message by one and calls the leave tree procedure for it.

\end{fulllineitems}

\index{on\_receiving\_become\_passive() (ShavitFrancez.ShavitFrancez.ShavitFrancezComponentModel method)@\spxentry{on\_receiving\_become\_passive()}\spxextra{ShavitFrancez.ShavitFrancez.ShavitFrancezComponentModel method}}

\begin{fulllineitems}
\phantomsection\label{\detokenize{docs/ShavitFrancez/generated/ShavitFrancez.ShavitFrancez:ShavitFrancez.ShavitFrancez.ShavitFrancezComponentModel.on_receiving_become_passive}}
\pysigstartsignatures
\pysiglinewithargsret{\sphinxbfcode{\sphinxupquote{on\_receiving\_become\_passive}}}{\sphinxparam{\DUrole{n}{eventobj}}}{}
\pysigstopsignatures
\sphinxAtStartPar
The process receiving the BECOMEPASSIVE event transitions to passive
state if not already passive, and calls the leave tree procedure.

\end{fulllineitems}

\index{on\_message\_from\_bottom() (ShavitFrancez.ShavitFrancez.ShavitFrancezComponentModel method)@\spxentry{on\_message\_from\_bottom()}\spxextra{ShavitFrancez.ShavitFrancez.ShavitFrancezComponentModel method}}

\begin{fulllineitems}
\phantomsection\label{\detokenize{docs/ShavitFrancez/generated/ShavitFrancez.ShavitFrancez:ShavitFrancez.ShavitFrancez.ShavitFrancezComponentModel.on_message_from_bottom}}
\pysigstartsignatures
\pysiglinewithargsret{\sphinxbfcode{\sphinxupquote{on\_message\_from\_bottom}}}{\sphinxparam{\DUrole{n}{eventobj}}}{}
\pysigstopsignatures
\sphinxAtStartPar
This method calls the related methods according to the message type of
the MFRT event.

\end{fulllineitems}

\index{send\_basic\_message() (ShavitFrancez.ShavitFrancez.ShavitFrancezComponentModel method)@\spxentry{send\_basic\_message()}\spxextra{ShavitFrancez.ShavitFrancez.ShavitFrancezComponentModel method}}

\begin{fulllineitems}
\phantomsection\label{\detokenize{docs/ShavitFrancez/generated/ShavitFrancez.ShavitFrancez:ShavitFrancez.ShavitFrancez.ShavitFrancezComponentModel.send_basic_message}}
\pysigstartsignatures
\pysiglinewithargsret{\sphinxbfcode{\sphinxupquote{send\_basic\_message}}}{}{}
\pysigstopsignatures
\sphinxAtStartPar
This method increases the number of children of the process
and makes the process send basic messages to its neighbors

\end{fulllineitems}

\index{send\_wave\_message() (ShavitFrancez.ShavitFrancez.ShavitFrancezComponentModel method)@\spxentry{send\_wave\_message()}\spxextra{ShavitFrancez.ShavitFrancez.ShavitFrancezComponentModel method}}

\begin{fulllineitems}
\phantomsection\label{\detokenize{docs/ShavitFrancez/generated/ShavitFrancez.ShavitFrancez:ShavitFrancez.ShavitFrancez.ShavitFrancezComponentModel.send_wave_message}}
\pysigstartsignatures
\pysiglinewithargsret{\sphinxbfcode{\sphinxupquote{send\_wave\_message}}}{}{}
\pysigstopsignatures
\sphinxAtStartPar
This method sends wave messages to the process’ all its neighbors

\end{fulllineitems}

\index{on\_receiving\_start\_wave() (ShavitFrancez.ShavitFrancez.ShavitFrancezComponentModel method)@\spxentry{on\_receiving\_start\_wave()}\spxextra{ShavitFrancez.ShavitFrancez.ShavitFrancezComponentModel method}}

\begin{fulllineitems}
\phantomsection\label{\detokenize{docs/ShavitFrancez/generated/ShavitFrancez.ShavitFrancez:ShavitFrancez.ShavitFrancez.ShavitFrancezComponentModel.on_receiving_start_wave}}
\pysigstartsignatures
\pysiglinewithargsret{\sphinxbfcode{\sphinxupquote{on\_receiving\_start\_wave}}}{\sphinxparam{\DUrole{n}{eventobj}}}{}
\pysigstopsignatures
\sphinxAtStartPar
This method implements the Echo algorithm on processes that
are currently not active and do not have any children.

\end{fulllineitems}

\index{generate\_message() (ShavitFrancez.ShavitFrancez.ShavitFrancezComponentModel method)@\spxentry{generate\_message()}\spxextra{ShavitFrancez.ShavitFrancez.ShavitFrancezComponentModel method}}

\begin{fulllineitems}
\phantomsection\label{\detokenize{docs/ShavitFrancez/generated/ShavitFrancez.ShavitFrancez:ShavitFrancez.ShavitFrancez.ShavitFrancezComponentModel.generate_message}}
\pysigstartsignatures
\pysiglinewithargsret{\sphinxbfcode{\sphinxupquote{generate\_message}}}{\sphinxparam{\DUrole{n}{messagetype}}\sphinxparamcomma \sphinxparam{\DUrole{n}{messageto}}}{}
\pysigstopsignatures
\end{fulllineitems}

\index{leave\_tree() (ShavitFrancez.ShavitFrancez.ShavitFrancezComponentModel method)@\spxentry{leave\_tree()}\spxextra{ShavitFrancez.ShavitFrancez.ShavitFrancezComponentModel method}}

\begin{fulllineitems}
\phantomsection\label{\detokenize{docs/ShavitFrancez/generated/ShavitFrancez.ShavitFrancez:ShavitFrancez.ShavitFrancez.ShavitFrancezComponentModel.leave_tree}}
\pysigstartsignatures
\pysiglinewithargsret{\sphinxbfcode{\sphinxupquote{leave\_tree}}}{}{}
\pysigstopsignatures
\sphinxAtStartPar
This method checks if the process is not currenty active and does not have any children.
According to that, if also the process has a parent, it sends acknowledge message to its
parent and leaves the parent’s tree. If the process does not have any parent, then it
starts a wave.

\end{fulllineitems}

\index{decide() (ShavitFrancez.ShavitFrancez.ShavitFrancezComponentModel method)@\spxentry{decide()}\spxextra{ShavitFrancez.ShavitFrancez.ShavitFrancezComponentModel method}}

\begin{fulllineitems}
\phantomsection\label{\detokenize{docs/ShavitFrancez/generated/ShavitFrancez.ShavitFrancez:ShavitFrancez.ShavitFrancez.ShavitFrancezComponentModel.decide}}
\pysigstartsignatures
\pysiglinewithargsret{\sphinxbfcode{\sphinxupquote{decide}}}{}{}
\pysigstopsignatures
\end{fulllineitems}

\index{announce() (ShavitFrancez.ShavitFrancez.ShavitFrancezComponentModel method)@\spxentry{announce()}\spxextra{ShavitFrancez.ShavitFrancez.ShavitFrancezComponentModel method}}

\begin{fulllineitems}
\phantomsection\label{\detokenize{docs/ShavitFrancez/generated/ShavitFrancez.ShavitFrancez:ShavitFrancez.ShavitFrancez.ShavitFrancezComponentModel.announce}}
\pysigstartsignatures
\pysiglinewithargsret{\sphinxbfcode{\sphinxupquote{announce}}}{}{}
\pysigstopsignatures
\end{fulllineitems}


\end{fulllineitems}

\index{ShavitFrancezEventTypes (class in ShavitFrancez.ShavitFrancez)@\spxentry{ShavitFrancezEventTypes}\spxextra{class in ShavitFrancez.ShavitFrancez}}

\begin{fulllineitems}
\phantomsection\label{\detokenize{docs/ShavitFrancez/generated/ShavitFrancez.ShavitFrancez:ShavitFrancez.ShavitFrancez.ShavitFrancezEventTypes}}
\pysigstartsignatures
\pysiglinewithargsret{\sphinxbfcode{\sphinxupquote{class\DUrole{w}{ }}}\sphinxcode{\sphinxupquote{ShavitFrancez.ShavitFrancez.}}\sphinxbfcode{\sphinxupquote{ShavitFrancezEventTypes}}}{\sphinxparam{\DUrole{n}{value}}\sphinxparamcomma \sphinxparam{\DUrole{n}{names=\textless{}not given\textgreater{}}}\sphinxparamcomma \sphinxparam{\DUrole{n}{*values}}\sphinxparamcomma \sphinxparam{\DUrole{n}{module=None}}\sphinxparamcomma \sphinxparam{\DUrole{n}{qualname=None}}\sphinxparamcomma \sphinxparam{\DUrole{n}{type=None}}\sphinxparamcomma \sphinxparam{\DUrole{n}{start=1}}\sphinxparamcomma \sphinxparam{\DUrole{n}{boundary=None}}}{}
\pysigstopsignatures\index{DETECTTERMINATION (ShavitFrancez.ShavitFrancez.ShavitFrancezEventTypes attribute)@\spxentry{DETECTTERMINATION}\spxextra{ShavitFrancez.ShavitFrancez.ShavitFrancezEventTypes attribute}}

\begin{fulllineitems}
\phantomsection\label{\detokenize{docs/ShavitFrancez/generated/ShavitFrancez.ShavitFrancez:ShavitFrancez.ShavitFrancez.ShavitFrancezEventTypes.DETECTTERMINATION}}
\pysigstartsignatures
\pysigline{\sphinxbfcode{\sphinxupquote{DETECTTERMINATION}}\sphinxbfcode{\sphinxupquote{\DUrole{w}{ }\DUrole{p}{=}\DUrole{w}{ }\textquotesingle{}DETECTTERMINATION\textquotesingle{}}}}
\pysigstopsignatures
\end{fulllineitems}

\index{BECOMEPASSIVE (ShavitFrancez.ShavitFrancez.ShavitFrancezEventTypes attribute)@\spxentry{BECOMEPASSIVE}\spxextra{ShavitFrancez.ShavitFrancez.ShavitFrancezEventTypes attribute}}

\begin{fulllineitems}
\phantomsection\label{\detokenize{docs/ShavitFrancez/generated/ShavitFrancez.ShavitFrancez:ShavitFrancez.ShavitFrancez.ShavitFrancezEventTypes.BECOMEPASSIVE}}
\pysigstartsignatures
\pysigline{\sphinxbfcode{\sphinxupquote{BECOMEPASSIVE}}\sphinxbfcode{\sphinxupquote{\DUrole{w}{ }\DUrole{p}{=}\DUrole{w}{ }\textquotesingle{}BECOMEPASSIVE\textquotesingle{}}}}
\pysigstopsignatures
\end{fulllineitems}

\index{SENDBASICMESSAGE (ShavitFrancez.ShavitFrancez.ShavitFrancezEventTypes attribute)@\spxentry{SENDBASICMESSAGE}\spxextra{ShavitFrancez.ShavitFrancez.ShavitFrancezEventTypes attribute}}

\begin{fulllineitems}
\phantomsection\label{\detokenize{docs/ShavitFrancez/generated/ShavitFrancez.ShavitFrancez:ShavitFrancez.ShavitFrancez.ShavitFrancezEventTypes.SENDBASICMESSAGE}}
\pysigstartsignatures
\pysigline{\sphinxbfcode{\sphinxupquote{SENDBASICMESSAGE}}\sphinxbfcode{\sphinxupquote{\DUrole{w}{ }\DUrole{p}{=}\DUrole{w}{ }\textquotesingle{}SENDBASICMESSAGE\textquotesingle{}}}}
\pysigstopsignatures
\end{fulllineitems}


\end{fulllineitems}

\index{ShavitFrancezMessageTypes (class in ShavitFrancez.ShavitFrancez)@\spxentry{ShavitFrancezMessageTypes}\spxextra{class in ShavitFrancez.ShavitFrancez}}

\begin{fulllineitems}
\phantomsection\label{\detokenize{docs/ShavitFrancez/generated/ShavitFrancez.ShavitFrancez:ShavitFrancez.ShavitFrancez.ShavitFrancezMessageTypes}}
\pysigstartsignatures
\pysiglinewithargsret{\sphinxbfcode{\sphinxupquote{class\DUrole{w}{ }}}\sphinxcode{\sphinxupquote{ShavitFrancez.ShavitFrancez.}}\sphinxbfcode{\sphinxupquote{ShavitFrancezMessageTypes}}}{\sphinxparam{\DUrole{n}{value}}\sphinxparamcomma \sphinxparam{\DUrole{n}{names=\textless{}not given\textgreater{}}}\sphinxparamcomma \sphinxparam{\DUrole{n}{*values}}\sphinxparamcomma \sphinxparam{\DUrole{n}{module=None}}\sphinxparamcomma \sphinxparam{\DUrole{n}{qualname=None}}\sphinxparamcomma \sphinxparam{\DUrole{n}{type=None}}\sphinxparamcomma \sphinxparam{\DUrole{n}{start=1}}\sphinxparamcomma \sphinxparam{\DUrole{n}{boundary=None}}}{}
\pysigstopsignatures\index{ACKNOWLEDGE (ShavitFrancez.ShavitFrancez.ShavitFrancezMessageTypes attribute)@\spxentry{ACKNOWLEDGE}\spxextra{ShavitFrancez.ShavitFrancez.ShavitFrancezMessageTypes attribute}}

\begin{fulllineitems}
\phantomsection\label{\detokenize{docs/ShavitFrancez/generated/ShavitFrancez.ShavitFrancez:ShavitFrancez.ShavitFrancez.ShavitFrancezMessageTypes.ACKNOWLEDGE}}
\pysigstartsignatures
\pysigline{\sphinxbfcode{\sphinxupquote{ACKNOWLEDGE}}\sphinxbfcode{\sphinxupquote{\DUrole{w}{ }\DUrole{p}{=}\DUrole{w}{ }\textquotesingle{}ACKNOWLEDGE\textquotesingle{}}}}
\pysigstopsignatures
\end{fulllineitems}

\index{BASICMESSAGE (ShavitFrancez.ShavitFrancez.ShavitFrancezMessageTypes attribute)@\spxentry{BASICMESSAGE}\spxextra{ShavitFrancez.ShavitFrancez.ShavitFrancezMessageTypes attribute}}

\begin{fulllineitems}
\phantomsection\label{\detokenize{docs/ShavitFrancez/generated/ShavitFrancez.ShavitFrancez:ShavitFrancez.ShavitFrancez.ShavitFrancezMessageTypes.BASICMESSAGE}}
\pysigstartsignatures
\pysigline{\sphinxbfcode{\sphinxupquote{BASICMESSAGE}}\sphinxbfcode{\sphinxupquote{\DUrole{w}{ }\DUrole{p}{=}\DUrole{w}{ }\textquotesingle{}BASICMESSAGE\textquotesingle{}}}}
\pysigstopsignatures
\end{fulllineitems}

\index{WAVE (ShavitFrancez.ShavitFrancez.ShavitFrancezMessageTypes attribute)@\spxentry{WAVE}\spxextra{ShavitFrancez.ShavitFrancez.ShavitFrancezMessageTypes attribute}}

\begin{fulllineitems}
\phantomsection\label{\detokenize{docs/ShavitFrancez/generated/ShavitFrancez.ShavitFrancez:ShavitFrancez.ShavitFrancez.ShavitFrancezMessageTypes.WAVE}}
\pysigstartsignatures
\pysigline{\sphinxbfcode{\sphinxupquote{WAVE}}\sphinxbfcode{\sphinxupquote{\DUrole{w}{ }\DUrole{p}{=}\DUrole{w}{ }\textquotesingle{}WAVE\textquotesingle{}}}}
\pysigstopsignatures
\end{fulllineitems}

\index{NOTIFYPROCESSES (ShavitFrancez.ShavitFrancez.ShavitFrancezMessageTypes attribute)@\spxentry{NOTIFYPROCESSES}\spxextra{ShavitFrancez.ShavitFrancez.ShavitFrancezMessageTypes attribute}}

\begin{fulllineitems}
\phantomsection\label{\detokenize{docs/ShavitFrancez/generated/ShavitFrancez.ShavitFrancez:ShavitFrancez.ShavitFrancez.ShavitFrancezMessageTypes.NOTIFYPROCESSES}}
\pysigstartsignatures
\pysigline{\sphinxbfcode{\sphinxupquote{NOTIFYPROCESSES}}\sphinxbfcode{\sphinxupquote{\DUrole{w}{ }\DUrole{p}{=}\DUrole{w}{ }\textquotesingle{}NOTIFYPROCESSES\textquotesingle{}}}}
\pysigstopsignatures
\end{fulllineitems}


\end{fulllineitems}


\begin{sphinxadmonition}{attention}{Attention:}
\sphinxAtStartPar
For RST details, please refer to \sphinxhref{https://docutils.sourceforge.io/rst.html}{reStructuredText Documentation}.
\end{sphinxadmonition}


\chapter{Indices and tables}
\label{\detokenize{index:indices-and-tables}}\begin{itemize}
\item {} 
\sphinxAtStartPar
\DUrole{xref,std,std-ref}{genindex}

\item {} 
\sphinxAtStartPar
\DUrole{xref,std,std-ref}{modindex}

\item {} 
\sphinxAtStartPar
\DUrole{xref,std,std-ref}{search}

\end{itemize}

\begin{sphinxthebibliography}{Dijkstra}
\bibitem[ShavitFrancez1986]{docs/ShavitFrancez/algorithm:shavitfrancez1986}
\sphinxAtStartPar
Shavit, N. and Francez, N. A new approach to the detection of locally indicative stability. In proc. Int. Colloq. Automata, Languages, and Programming (1986), L. Kott (ed.), vol. 226 of Lecture Notes in Computer Science, Springer\sphinxhyphen{}Verlag, pp. 344\sphinxhyphen{}358.
\bibitem[Fokking2013]{docs/ShavitFrancez/algorithm:fokking2013}
\sphinxAtStartPar
Wan Fokkink, Distributed Algorithms An Intuitive Approach, The MIT Press Cambridge, Massachusetts London, England, 2013
\bibitem[DijkstraSholten1980]{docs/ShavitFrancez/algorithm:dijkstrasholten1980}
\sphinxAtStartPar
Dijkstra, E. W. and Scholten, C. S. Termination detection for diffusing computations. Inf. Proc. Lett. 11, 1 (1980), 1\sphinxhyphen{}4.
\bibitem[Tel2001]{docs/ShavitFrancez/algorithm:tel2001}
\sphinxAtStartPar
Tel, G, Introduction To Distributed Algorithms, The Cambridge University Press, Cambridge, United Kingdom, 2001
\end{sphinxthebibliography}


\renewcommand{\indexname}{Python Module Index}
\begin{sphinxtheindex}
\let\bigletter\sphinxstyleindexlettergroup
\bigletter{s}
\item\relax\sphinxstyleindexentry{ShavitFrancez.ShavitFrancez}\sphinxstyleindexpageref{docs/ShavitFrancez/generated/ShavitFrancez.ShavitFrancez:\detokenize{module-ShavitFrancez.ShavitFrancez}}
\end{sphinxtheindex}

\renewcommand{\indexname}{Index}
\printindex
\end{document}